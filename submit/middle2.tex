\documentclass{jarticle}
\usepackage[dvipdfmx]{graphicx}
\usepackage{here}
\usepackage{listings}

\begin{document}

\title{中間レポート2}
\author{1029289895 尾崎翔太}
\date{2018/11/01}

\maketitle
\newpage

\section{課題6}
\subsection{プログラムの設計方針}
まず, 木の型がvalueとcexpとexpに分かれてしまったのでそれぞれ変換する関数を用意する. 関数の引数にクロージャを追加したりinの後でクロージャを束縛したりすることは, 単にそこを書き換えたり追加したりするだけなので簡単である. 問題は自由変数を含んでいた時の本体式の書き換えである. まず, 自由変数を求めるために式中の変数を集めて, その中から束縛されているものを取り除く必要がある. そうした後で, 自由変数の値をクロージャから取り出して, 本体式の頭で束縛していく必要がある. この部分は正規化で行ったのと同じ埋め込みを行う. この時, 埋め込みに使う関数は合成を用いて作ることにした.
\subsection{プログラムの説明}
closure.mlに定義されている関数について説明する.
\subsubsection{cvalue\_of\_nvalue関数, ccexp\_of\_ncexp関数(113〜127行目)}
ほぼ単純にコンストラクタを付け替えていく関数である. IfExpはexpをぶら下げていて特殊なので, この関数では処理しない. また, AppExpについては, 関数ポインタをクロージャから取り出すという作業を追加している. 
\subsubsection{gather\_id\_from\_XXX関数(132〜166行目)}
expを巡回して, 束縛されている変数を見つけるとそのidをdecl\_idsへ, 参照されている変数を見つけるとそのidをvar\_idsへ放り込んでいく関数である. 
\subsubsection{delete\_duplication関数(169〜173行目)}
リスト内の重複を取り除くための関数で, gather\_id\_from\_XXX関数で集めてきたidの重複を取り除くために使われている. 
\subsubsection{remove関数(177〜181行目)}
リストからある要素を一つ取り除く関数である. この関数は後述するdiff関数のためにある. 
\subsubsection{diff関数(184〜188行目)}
差集合を求める関数である. これにより, 参照されている変数から束縛された変数を引くことで自由変数が求まる. 
\subsubsection{var\_of\_id関数(191〜193行目)}
idにvalue型のコンストラクタVarを付けてvalue型にする関数である. クロージャはid型ではなくvalue型を用いるので, そのためである.
\subsubsection{add\_to\_closure関数(196行目)}
クロージャに引数を追加する関数である.
\subsubsection{convert関数(201〜256行目)}
実際にクロージャ変換を行う関数である. expを巡回して, N.を外していきながらLetRecExpを見つけるとクロージャ変換を行う. クロージャ変換についてはソースコード中にコメントを付けたので, そちらを参照していただきたい.
\subsection{感想}
一番大変だったのはletrec式の本体式で自由変数をクロージャから取り出して束縛していくことだった. 正規化と違い, let式の中にlet式を作っていく感じになって難しかったのだが, このレポートを書いている時に, 「自由変数のリストを反転すればlet式の外にlet式を作っていく感じになって正規化と同じようにできたのではないか」と思った. また, IfExpの扱いについては, ccexp\_of\_ncexp関数をbody\_loop関数と並べて書けばあまり特別扱いする必要はなかったのだが, IfExpのときにしか用いないfをccexp\_of\_ncexp関数に渡すのはどうかと思って切り分けた.

\section{課題7}
\subsection{プログラムの設計方針}
資料に書いてあるように実装したので, あまり特筆すべきことはない.
\subsection{プログラムの説明}
flat.mlに定義されている関数について説明する.
\subsubsection{fvalue\_of\_cvalue関数(118〜125行目)}
Closure.value型をFlat.value型に変換する関数である. 環境も引数にとって, VarかFunかを決める. また, 束縛されていない変数への参照に対するエラーも吐く.
\subsubsection{fvalue\_list\_of\_cvalue\_list関数(128〜131行目)}
fvalue\_of\_cvalue関数のリスト版である. AppExpやTupleExpがvalue listをぶら下げているので必要となる関数である.
\subsubsection{fcexp\_of\_ccexp関数(134〜141行目)}
Closure.cexp型をFlat.cexp型に変換する関数である. 相変わらずIfExpは処理しない.
\subsubsection{extend\_env\_by\_id\_list関数(144〜149行目)}
リストの中身をVarとして環境に追加していく関数である. letrec式が複数引数をとるようになったので, そこで使う.
\subsubsection{flatten関数(153〜194行目)}
実際に平滑化を行う関数である. 資料にあったようにexpを巡回してLetRecExpをdecl\_listに放り込んでいっている. 環境の扱いも資料の通りである. 資料にないものとして, letrec式があったところを, そのletrec式のin以下にぶら下がっているexpに置き換えることでtop\_level関数の本体式を構成している. 
\subsection{感想}
簡単という割にはtop\_level関数の作り方で少し悩んでしまった. 今回はfcexp\_of\_ccexp関数をbody\_loop関数と一緒に定義してIfExpも一緒くたにしても良いと思ったが, closure.mlと揃えることにした. 

\end{document}
